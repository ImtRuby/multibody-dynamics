\documentclass{beamer}

%%%%%%%%%%%%%%%%%%%%%%%%%%%%%%%%%%%%%%%%%%%%%%%%%%%%%%%%%%%%%%%%
%%%                  Themes and such                         %%%
%%%%%%%%%%%%%%%%%%%%%%%%%%%%%%%%%%%%%%%%%%%%%%%%%%%%%%%%%%%%%%%%
\mode<presentation>
{
  %\usetheme{Copenhagen}  
  %\usetheme{Warsaw}  
  \usetheme{Malmoe}  
%    \setbeamertemplate{headline}{}
  %make my huge toc fit on one slide (and not look horrible)
  %\setbeamerfont{subsection in toc}{series=\bfseries}
  %\setbeamerfont{subsection in toc}{size=\tiny,series=\bfseries}
}

%%%%%%%%%%%%%%%%%%%%%%%%%%%%%%%%%%%%%%%%%%%%%%%%%%%%%%%%%%%%%%%%
%%%                       Packages                           %%%
%%%%%%%%%%%%%%%%%%%%%%%%%%%%%%%%%%%%%%%%%%%%%%%%%%%%%%%%%%%%%%%%
\usepackage{multimedia}
\usepackage{multirow}
\usepackage{subfigure}
\usepackage{amsmath}

% Define commands
 \newcommand{\half}{\ensuremath{\frac{1}{2}}}

 \newcommand{\bea}{\begin{eqnarray}}
 \newcommand{\eea}{\end{eqnarray}}
 \newcommand{\beq}{\begin{equation}}
 \newcommand{\eeq}{\end{equation}}
 \newcommand{\bed}{\begin{displaymath}}
 \newcommand{\eed}{\end{displaymath}}

 \newcommand{\pd}[2]{\frac{\partial #1}{\partial #2}}
 \newcommand{\pf}[2]{\frac{d #1}{d #2}}
 \newcommand{\pdt}[2]{\frac{\partial^2 #1}{\partial #2^2}}
 \newcommand{\pft}[2]{\frac{d^2 #1}{d #2^2}}
 \newcommand{\pdtno}[2]{\frac{\partial^2 #1}{\partial #2}}
 \newcommand{\pdd}[3]{\frac{\partial^2 #1}{\partial #2 \partial #3}}
 \newcommand{\pff}[3]{\frac{d^2 #1}{d #2 d #3}}

 \graphicspath{{../figures/}}


%%%%%%%%%%%%%%%%%%%%%%%%%%%%%%%%%%%%%%%%%%%%%%%%%%%%%%%%%%%%%%%%
%%%                     Title Info                           %%%
%%%%%%%%%%%%%%%%%%%%%%%%%%%%%%%%%%%%%%%%%%%%%%%%%%%%%%%%%%%%%%%%

\title[\hspace{-0.2cm} DIRK Adjoint]
{
Discrete Adjoint: Adams--Bashforth--Moulton (ABM)
}

\author[Komahan Boopathy]
{
  \Large {Komahan Boopathy}\\
}

\institute
{
  \large Georgia Institute of Technology\\
 School of Aerospace Engineering\\
 Atlanta, GA
}

\date
{
\small \today
}

\begin{document}

\begin{frame}
  \titlepage
\end{frame}

%\begin{frame}
%  \frametitle{Outline}
%  \tableofcontents
%\end{frame}

\begin{frame}[allowframebreaks] \frametitle{Time Integration}

  The second--order governing differential equations are posed in the following
  descriptor form at the k-th time step:
  $$ R_k(\ddot{q}_k, \dot{q}_k, q_k, t_k , x) = 0.$$

  We use Adams--Bashforth--Moulton (ABM) method to approximate the states:
  $$ \dot{q}_k  = \dot{q}_{k-1}  + h \sum_{i=1}^{m} a_i \ddot{q}_{k-i+1}$$ 
  $$ {q}_k      =     {q}_{k-1}  + h \sum_{i=1}^{m} a_i \dot{q}_{k-i+1}.$$

  \framebreak

  The ABM coefficients upto $6^{th}$ order are tabulated.

  \begin{center}
    \begin{tabular}{cccccccc}
      \hline
      k & $i=0$ & 1 & 2 & 3 & 4 & 5 & 6 \\ 
      \hline
      &&&&&&&\\
      1 & 1     &   &   &   &   &   &  \\
      &&&&&&&\\
      2 & $\frac{1}{2}$ & $\frac{1}{2}$ &   &   &   &   &  \\
      &&&&&&&\\
      3 & $\frac{5}{12}$ & $\frac{8}{12}$ & $-\frac{1}{12}$  &   &   &   &  \\
      &&&&&&&\\
      4 & $\frac{9}{24}$ & $\frac{19}{24}$ & $-\frac{5}{24}$  & $\frac{1}{24}$  &   &   &  \\
      &&&&&&&\\
      5 & $\frac{251}{720}$ & $\frac{646}{720}$ & $-\frac{264}{720}$  & $\frac{106}{720}$  & $-\frac{19}{720}$  &   &  \\
      &&&&&&&\\
      6 & $\frac{475}{1440}$ & $\frac{1427}{1440}$ & $-\frac{798}{1440}$  & $\frac{482}{1440}$  & $-\frac{173}{1440}$  &  $\frac{27}{1440}$ &  \\
      &&&&&&&\\
      \hline   
    \end{tabular}
  \end{center}

  \framebreak

  We first demonstrate the adjoint on a three-step time integation. \\

  First step $(k=1, m=1)$: $$R_1(\underline{\ddot{q}_1}, \dot{q}_1, q_1, t_1) = 0$$
  where
  $$ \dot{q}_1  = \dot{q}_{0}  + h a_1 \ddot{q}_{1}$$ 
  $$     {q}_1  =     {q}_{0}  + h a_1 \dot{q}_{1}$$ 

  Second step $(k=2, m=2)$: $$R_2(\underline{\ddot{q}_2}, \dot{q}_2, q_2, t_2) = 0$$
  where
  $$ \dot{q}_2  = \dot{q}_{1}  + h a_1 \ddot{q}_{2} + h a_2 \ddot{q}_{1} $$ 
  $$     {q}_2  =     {q}_{1}  + h a_1 \dot{q}_{2}  + h a_2 \dot{q}_{1} $$ 

  We introduce $\lambda$ as adjoint variables and form the Lagrangian: $$ {\cal L} = \sum_{k=0}^N f_k + \sum_{k=0}^N \lambda_k^T R_k $$

  We set $\partial{\cal{L}}/\partial{\ddot{q}_2}=0$ to solve for $\lambda_2$.
  
  $$\pd{f_2}{\ddot{q}_2} + \lambda_2^T \pd{R_2}{\ddot{q}_2} = 0$$

  Expanding and transposing yields:
  $$
  \left[ \pd{R_2}{\ddot{q}_2} + ha_1 \pd{R_2}{\dot{q}_2} + h^2a_1^2 \pd{R_2}{{q}_2} \right]^T \lambda_2 = - \left\{ \pd{f_2}{\ddot{q}_2} + ha_1 \pd{f_2}{\dot{q}_2} + h^2a_1^2 \pd{f_2}{{q}_2} \right\}^T 
  $$
  
  We set $\partial{\cal{L}}/\partial{\ddot{q}_1}=0$ to solve for $\lambda_1$. 
  
  $$\pd{f_1}{\ddot{q}_1} + \lambda_1^T \pd{R_1}{\ddot{q}_1} 
  + \pd{f_2}{\ddot{q}_1} + \lambda_2^T \pd{R_2}{\ddot{q}_1} = 0$$

\begin{equation}\nonumber
\begin{split}
  \left[ \pd{R_1}{\ddot{q}_1} + ha_1 \pd{R_1}{\dot{q}_1} + h^2a_1^2 \pd{R_1}{{q}_1} \right]^T \lambda_1  = &
  - \left\{ \pd{f_1}{\ddot{q}_1} + ha_1 \pd{f_1}{\dot{q}_1} + h^2a_1^2 \pd{f_1}{{q}_1} \right\}^T \\
  & - \left\{ \pd{f_2}{\dot{q}_2} \pd{\dot{q}_2}{\ddot{q}_1} +  \pd{f_2}{{q}_2} \pd{{q}_2}{\ddot{q}_1} \right\}^T\\
  & - \left[  \pd{R_2}{\dot{q}_2} \pd{\dot{q}_2}{\ddot{q}_1} +  \pd{R_2}{{q}_2} \pd{{q}_2}{\ddot{q}_1} \right]^T \lambda_1 \\
\end{split}
\end{equation}
where
$$\alpha=\pd{q_2}{\ddot{q}_1} = h^2a_1^2 + ha_2(ha_1) + ha_1(ha_1+ha_1)$$
$$
\beta=\pd{\dot{q}_2}{\ddot{q}_1} = ha_2
$$
\end{frame}

\end{document}

