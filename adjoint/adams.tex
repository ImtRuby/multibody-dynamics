\documentclass{beamer}

%%%%%%%%%%%%%%%%%%%%%%%%%%%%%%%%%%%%%%%%%%%%%%%%%%%%%%%%%%%%%%%%
%%%                  Themes and such                         %%%
%%%%%%%%%%%%%%%%%%%%%%%%%%%%%%%%%%%%%%%%%%%%%%%%%%%%%%%%%%%%%%%%
\mode<presentation>
{
  %\usetheme{Copenhagen}  
  %\usetheme{Warsaw}  
  \usetheme{Malmoe}  
%    \setbeamertemplate{headline}{}
  %make my huge toc fit on one slide (and not look horrible)
  %\setbeamerfont{subsection in toc}{series=\bfseries}
  %\setbeamerfont{subsection in toc}{size=\tiny,series=\bfseries}
}

%%%%%%%%%%%%%%%%%%%%%%%%%%%%%%%%%%%%%%%%%%%%%%%%%%%%%%%%%%%%%%%%
%%%                       Packages                           %%%
%%%%%%%%%%%%%%%%%%%%%%%%%%%%%%%%%%%%%%%%%%%%%%%%%%%%%%%%%%%%%%%%
\usepackage{multimedia}
\usepackage{multirow}
\usepackage{subfigure}
\usepackage{amsmath}

% Define commands
 \newcommand{\half}{\ensuremath{\frac{1}{2}}}

 \newcommand{\bea}{\begin{eqnarray}}
 \newcommand{\eea}{\end{eqnarray}}
 \newcommand{\beq}{\begin{equation}}
 \newcommand{\eeq}{\end{equation}}
 \newcommand{\bed}{\begin{displaymath}}
 \newcommand{\eed}{\end{displaymath}}

 \newcommand{\pd}[2]{\frac{\partial #1}{\partial #2}}
 \newcommand{\pf}[2]{\frac{d #1}{d #2}}
 \newcommand{\pdt}[2]{\frac{\partial^2 #1}{\partial #2^2}}
 \newcommand{\pft}[2]{\frac{d^2 #1}{d #2^2}}
 \newcommand{\pdtno}[2]{\frac{\partial^2 #1}{\partial #2}}
 \newcommand{\pdd}[3]{\frac{\partial^2 #1}{\partial #2 \partial #3}}
 \newcommand{\pff}[3]{\frac{d^2 #1}{d #2 d #3}}

 \graphicspath{{../figures/}}


%%%%%%%%%%%%%%%%%%%%%%%%%%%%%%%%%%%%%%%%%%%%%%%%%%%%%%%%%%%%%%%%
%%%                     Title Info                           %%%
%%%%%%%%%%%%%%%%%%%%%%%%%%%%%%%%%%%%%%%%%%%%%%%%%%%%%%%%%%%%%%%%

\title[\hspace{-0.2cm} DIRK Adjoint]
{
Discrete Adjoint: Adams--Bashforth--Moulton (ABM)
}

\author[Komahan Boopathy]
{
  \Large {Komahan Boopathy}\\
}

\institute
{
  \large Georgia Institute of Technology\\
 School of Aerospace Engineering\\
 Atlanta, GA
}

\date
{
\small \today
}

\begin{document}

\begin{frame}
  \titlepage
\end{frame}

%\begin{frame}
%  \frametitle{Outline}
%  \tableofcontents
%\end{frame}

\begin{frame}[allowframebreaks] \frametitle{ABM Adjoint}

  \tiny{

    The second--order governing differential equations are posed in the following
    descriptor form at the k-th time step:
    $$ R_k(\underline{\ddot{q}_k}, \dot{q}_k, q_k, t_k , x) = 0.$$

    We use Adams--Bashforth--Moulton (ABM) method to approximate the states:
    $$ S_k =   \dot{q}_{k-1}  + h \sum_{i=0}^{m-1} a_i \ddot{q}_{k-i} - \underline{\dot{q}_k}  = 0 $$ 
    $$ T_k =   {q}_{k-1}  + h \sum_{i=0}^{m-1} a_i \dot{q}_{k-i} - \underline{{q}_k} = 0.$$

    The acceleration states $\ddot{q}_k$ are the primary unknown at each time step. We introduce $\lambda_k$, $\psi_k$ and $\phi_k$ as adjoint variables
    associated with each of these equations. The Lagrangian function is written
    as:
    
    $${\cal{L}} = \sum_{k=0}^N h f_k + \sum_{k=0}^N h \lambda_k^T
    R_k + \sum_{k=0}^N \psi_k^T S_k + \sum_{k=0}^N \phi_k^T T_k. $$
    
    The underlined variables denote the primary unknown in each
    equation.
    
    \framebreak

    % Differentiate wrt to q
    
    $\star$ Setting $\pd{\cal{L}}{{q}_k} = 0$ yields:
    $$ \phi_{k}^T \pd{T_k}{q_k} + \phi_{k+1}^T \pd{T_{k+1}}{q_k} + h \lambda_{k+1}^T \pd{R_{k+1}}{q_{k}}
    + h \pd{f_{k+1}}{q_{k}}
    = 0 $$ which simplifies to
    
    \begin{center}
      \boxed{
        $$
        \phi_k = \phi_{k+1} + h \left[  \pd{R_{k+1}}{q_{k+1}} \right]^T \lambda_{k+1} + h \left\{ \pd{f_{k+1}}{q_{k+1}} \right\}^T
        $$
      }
    \end{center}

    % Differentiate wrt to qdot
    
    $\star$ Setting $\pd{\cal{L}}{\dot{q}_k} = 0$ yields:

    $$
    \psi_k^T \pd{S_k}{\dot{q}_k} + \psi_{k+1}^T \pd{S_{k+1}}{\dot{q}_k} +
    \sum_{i=0}^{m-1} \phi_{k+i}^T \pd{T_{k+i}}{\dot{q}_k}  + h \sum_{i=1}^{m-1} \lambda_{k+i}^T \pd{R_{k+i}}{\dot{q}_k} 
    + h \sum_{i=1}^{m-1} \pd{f_{k+i}}{\dot{q}_k} = 0
    $$

    which simplifies to

    \begin{equation}
      \begin{split}
        \psi_k = \psi_{k+1} + h \sum_{i=0}^{m-1} a_i \phi_{k+i} & + h \sum_{i=1}^{m-1}
        \left[ \pd{R_{k+i}}{\dot{q}_{k+i}} + \sum_{j=0}^{i} ha_j \pd{R_{k+i}}{{q}_{k+i}}  \right]^T \lambda_{k+1}\\
        & + h \sum_{i=1}^{m-1}
        \left\{ \pd{f_{k+i}}{\dot{q}_{k+i}} + \sum_{j=0}^{i} ha_j \pd{f_{k+i}}{{q}_{k+i}}  \right\}^T
      \end{split}
    \end{equation}

    % Differentiate wrt to qddot
    
    $\star$ Setting $\pd{\cal{L}}{\ddot{q}_k} = 0$ yields:

    $$ h \lambda_k^T \pd{R_k}{\ddot{q}_k} + h \pd{f_k}{\ddot{q}_k} +
    \sum_{i=1}^{m-1} h \lambda_{K+i} \pd{R_{k+i}}{\ddot{q}_k} + \sum_{i=1}^{m-1} h
    \pd{f_{k+1}}{\ddot{q}_k} + \sum_{i=0}^{m-1} \psi_{k+i}^T
    \pd{S_{k+i}}{\ddot{q}_k} =0
    $$
    
    which simplifies to
    
    \begin{equation}
      \begin{split}
        h \left[ \pd{R_k}{\ddot{q}_k} + h a_0 \pd{R_k}{\dot{q}_k} +  h^2 a_0^2  \pd{R_k}{{q}_k} \right]^T \lambda_k = & -  h \left\{ \pd{R_k}{\ddot{q}_k} + h a_0 \pd{R_k}{\dot{q}_k} +  h^2 a_0^2  \pd{R_k}{{q}_k} \right\}^T\\
        & - h \sum_{i=0}^{m-1} a_i \psi_{k+i} \\
        & - h \sum_{i=1}^{m-1} h \left[ \pd{R_{k+i}}{\dot{q}_{k+i}}\pd{\dot{q}_{k+i}}{\ddot{q}_k} + \pd{R_{k+i}}{q_{k+i}}\pd{q_{k+i}}{\ddot{q}_k} \right]^T \lambda_{k+i}\\
        & - h \sum_{i=1}^{m-1} h \left\{ \pd{f_{k+i}}{\dot{q}_{k+i}}\pd{\dot{q}_{k+i}}{\ddot{q}_k} + \pd{f_{k+i}}{q_{k+i}}\pd{q_{k+i}}{\ddot{q}_k} \right\}^T
      \end{split}
    \end{equation}

    \framebreak
    
    \begin{equation}
      \begin{split}
        h \left[ \pd{R_k}{\ddot{q}_k} + h a_0 \pd{R_k}{\dot{q}_k} +  h^2 a_0^2  \pd{R_k}{{q}_k} \right]^T \lambda_k = & -  h \left\{ \pd{R_k}{\ddot{q}_k} + h a_0 \pd{R_k}{\dot{q}_k} +  h^2 a_0^2  \pd{R_k}{{q}_k} \right\}^T\\
        & - h \sum_{i=0}^{m-1} a_i \psi_{k+i} \\
        & - h \sum_{i=1}^{m-1} h \left[ ha_i \pd{R_{k+i}}{\dot{q}_{k+i}} + \pd{R_{k+i}}{q_{k+i}} \left( \pd{q_{k+i}}{q_{k+i-1}}  \pd{q_{k+i-1}} {\ddot{q}_k} +  \pd{q_{k+i}}{\dot{q}_{k}} \pd{\dot{q}_{k}} {\ddot{q}_k} \right) \right]^T \lambda_{k+i}\\
        & - h \sum_{i=1}^{m-1} h \left\{ \pd{f_{k+i}}{\dot{q}_{k+i}}\pd{\dot{q}_{k+i}}{\ddot{q}_k} + \pd{f_{k+i}}{q_{k+i}}\pd{q_{k+i}}{\ddot{q}_k} \right\}^T
      \end{split}
    \end{equation}

  } % end tiny font
  
\end{frame}

\end{document}
%
%
%
\framebreak

The ABM coefficients upto $6^{th}$ order are tabulated.

\begin{center}
  \begin{tabular}{cccccccc}
    \hline
    m & $i=0$ & 1 & 2 & 3 & 4 & 5 & 6 \\ 
    \hline
    &&&&&&&\\
    1 & 1     &   &   &   &   &   &  \\
    &&&&&&&\\
    2 & $\frac{1}{2}$ & $\frac{1}{2}$ &   &   &   &   &  \\
    &&&&&&&\\
    3 & $\frac{5}{12}$ & $\frac{8}{12}$ & $-\frac{1}{12}$  &   &   &   &  \\
    &&&&&&&\\
    4 & $\frac{9}{24}$ & $\frac{19}{24}$ & $-\frac{5}{24}$  & $\frac{1}{24}$  &   &   &  \\
    &&&&&&&\\
    5 & $\frac{251}{720}$ & $\frac{646}{720}$ & $-\frac{264}{720}$  & $\frac{106}{720}$  & $-\frac{19}{720}$  &   &  \\
    &&&&&&&\\
    6 & $\frac{475}{1440}$ & $\frac{1427}{1440}$ & $-\frac{798}{1440}$  & $\frac{482}{1440}$  & $-\frac{173}{1440}$  &  $\frac{27}{1440}$ &  \\
    &&&&&&&\\
    \hline   
  \end{tabular}
\end{center}

\framebreak

We demonstrate the adjoint development for three steps of time
integation. \\
~\\
\underline{\textbf{First step}} $(k=1, m=1)$: 
$$ h \lambda_1^T R_1(\underline{\ddot{q}_1}, \dot{q}_1, q_1, t_1) = 0$$
$$ h  f_1(\underline{\ddot{q}_1}, \dot{q}_1, q_1, t_1)^T = 0 $$
$$ \psi_1^T S_1 = \dot{q}_{0}  + h a_0 \ddot{q}_{1} - \dot{q}_1  $$ 
$$ \phi_1^T T_1 =  {q}_{0}  + h a_0 \dot{q}_{1} - {q}_1$$ 

\underline{\textbf{Second step}} $(k=2, m=2)$: 
$$h \lambda_2^T R_2(\underline{\ddot{q}_2}, \dot{q}_2, q_2, t_2) = 0$$
$$h f_2(\underline{\ddot{q}_2}, \dot{q}_2, q_2, t_2)^T = 0$$
$$ \psi_2^T S_2 = \dot{q}_{1}  + h a_0 \ddot{q}_{2} + h a_1 \ddot{q}_{1} - \dot{q}_2 $$ 
$$ \phi_2^T T_2 = {q}_{1}  + h a_0 \dot{q}_{2}  + h a_1 \dot{q}_{1} - {q}_2$$ 

\underline{\textbf{Third step}} $(k=3, m=3)$: 
$$h\lambda_3^TR_3(\underline{\ddot{q}_3}, \dot{q}_3, q_3, t_3) = 0$$
$$hf_3(\underline{\ddot{q}_3}, \dot{q}_3, q_3, t_3)^T = 0$$
$$ \psi_3^TS_3 = \dot{q}_{2}  + h a_0 \ddot{q}_{3} + h a_1 \ddot{q}_{2}  + h a_2 \ddot{q}_{1} - \dot{q}_3 $$ 
$$ \phi_3^TT_3 = {q}_{2}  + h a_0 \dot{q}_{3} + h a_1 \dot{q}_{2}  + h a_2 \dot{q}_{1} - {q}_3 $$ 

